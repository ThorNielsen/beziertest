\documentclass[11pt, a4paper]{article}
\usepackage[utf8]{inputenc}
\usepackage[margin=25mm]{geometry}
\PassOptionsToPackage{usenames,dvipsnames,svgnames,table}{xcolor}
\usepackage[british]{babel}
\usepackage{fancyhdr}
\usepackage{amsmath}
\usepackage{amsfonts}
\usepackage{amssymb}
\usepackage{bm}
\usepackage{graphicx}
\usepackage{float}
\usepackage{titling}
\usepackage{centernot}
\usepackage{mathrsfs}
\usepackage{mathtools}
\usepackage{bbm}
\usepackage{halloweenmath}
\usepackage{listings}

\usepackage{hyperref}
\hypersetup{%
pdfauthor = {},
pdfborder = {0 0 0},
pdfcreator = {},
pdfproducer = {},
pdfstartpage = {1},
pdfsubject = {},
pdftitle = {}}

\usepackage{xcolor}
\usepackage{pgfornament}


\newcommand{\nimplies}{\centernot\implies}
\newcommand{\niton}{\centernot\ni}
\newcommand{\qedsymb}{\ensuremath{\blacksquare}}
\newcommand{\numberhere}{\stepcounter{equation}\tag{\theequation}}

\newlength{\listingindent}
\setlength{\listingindent}{1mm}
\font\tt=rm-lmtl10
\font\itt=rm-lmtlo10
\font\btt=rm-lmtk10
\font\bitt=rm-lmtko10
\renewcommand{\thelstnumber}
{
    \protect{\tt\arabic{lstnumber}}
}
\lstdefinestyle{cstyle}
{
    basicstyle=\tt,
    breaklines=true,
    captionpos=b,
    commentstyle=\color{gray},
    extendedchars=true,
    frame=Lrtb,
    identifierstyle=\color{black},
    keywordstyle=\btt\color[HTML]{002CAE},
    numbers=left,
    stringstyle=\color[HTML]{0037FF},
    showstringspaces=false,
    xleftmargin=2\listingindent,
    xrightmargin=1\listingindent,
}
\newcommand{\code}[1]{{\tt #1}}
\newcommand{\keyword}[1]{{\btt\color[HTML]{002CAE}#1}}

\renewcommand\vec[1]{\boldsymbol{#1}}

\begin{document}
In the following we will derive how the intersection count between a quadratic Bézier curve and a ray can be calculated efficiently and accurately.
\section{Definitions}
Let $\vec p_0 = (x_0, y_0)$ and $\vec p_2 = (x_2, y_2)$ denote the endpoints of the quadratic Bézier curve and let $\vec p_1 = (x_1, y_1)$ denote the control point. The curve can be written parametrically as
\begin{equation*}
Q(t) = (1-t)^2 \vec p_0 + 2(1-t)t \vec p_1 + t^2 \vec p_2 \qquad \text{ for } t \in (0, 1)
\end{equation*}
which can be rearranged to get
\begin{equation}
\label{eq:bezierdef}
Q(t) = (\vec p_2 - 2\vec p_1 + \vec p_0)t^2 + 2(\vec p_1 - \vec p_0)t + \vec p_0 \qquad \text{ for } t \in (0, 1)
\end{equation}
Furthermore we have a ray at position $\vec r = (r_x, r_y)$ with direction parallel to the $x$-axis, i.e.:
\begin{equation}
\label{eq:raydef}
R(s) = \vec r + (1, 0)^\top s \qquad \text{ for } s \geq 0
\end{equation}

\subsection{Problem}
Given the points of the quadratic Bézier curve along with the origin of the ray, determine the number of intersections between these two. We note that $\vec p_0$, $\vec p_1$ and $\vec p_2$ always has integer coordinates (with magnitude $\approx 2^{12}$) while $\vec r$'s coordinates can be any real numbers (again with magnitude $\approx 2^{12}$).

\section{Actual derivation}
The ray and Bézier curve intersect in all points where $Q(t) = R(s)$ for $(t, s)\in(0,1)\times[0,\infty)$.
This specifically means that in these points, $Q_y(t) = R_y(s)$. Inserting the definitions we get:
\begin{align*}
(y_2 - 2y_1 + y_0)t^2 + 2(y_1 - y_0)t + y_0 &= r_y &\iff\\
(y_2 - 2y_1 + y_0)t^2 + 2(y_1 - y_0)t + y_0 - r_y &= 0 \numberhere\label{eq:sdeq}
\end{align*}
which is a second-degree equation. Set $A=y_2-2y_1+y_0$, $B=2(y_1-y_0)$ and $C=y_0-r_y$.
We note that the number of solutions here can be $0, 1, 2$ or $\infty$ (the last one in the degenerate case where $A=B=C=0$).

\emph{In the following it is assumed that $A \neq 0$ unless otherwise stated.}

We can find the number of solutions by examining the discriminant:
\begin{align*}
D &= B^2-4AC\\
&= 2^2(y_1 - y_0)^2-4(y_2 - 2y_1 + y_0)(y_0 - r_y)\\
&= 4\left(y_1^2 +y_0^2-2y_0y_1-(y_0y_2 - 2y_0y_1 + y_0^2- r_y(y_2 - 2y_1 + y_0)\right)\\
&= 4\left(y_1^2 - y_0y_2 + r_y(y_2 - 2y_1 + y_0)\right)
\end{align*}
There are two solutions iff $D > 0$, i.e. iff
\begin{align*}
4\left(y_1^2 - y_0y_2 + r_y(y_2 - 2y_1 + y_0)\right) &> 0&\iff\\
y_1^2 - y_0y_2 &> -r_y(y_2 - 2y_1 + y_0)&\iff\\
y_0y_2-y_1^2 &< Ar_y\numberhere\label{eq:posdisc}
\end{align*}
If the inequality is flipped or replaced with equality, we obtain the conditions where $D < 0$ (giving no solutions) and $D = 0$ (resulting in a single solution), respectively.

If there are any solutions, they can be written as
\begin{equation}
\label{eq:ysolformula}
t_m = \frac{-B-\sqrt{D}}{2A}\qquad\text{and}\qquad t_p = \frac{-B+\sqrt{D}}{2A},
\end{equation}
possibly with $t_m = t_p$.

Now assume we have $D \geq 0$ (corresponding to two solutions to \eqref{eq:sdeq}). We wish to find the number of solutions which corresponds to $t \in (0, 1)$ in order be able to satisfy \eqref{eq:bezierdef}. However, it is easier to ignore the restriction $t<1$ for a moment and just find the number of solutions where $t > 0$. We simply check whether $t_m$ and $t_p$ are positive:
\begin{align*}
t_m &> 0 & \iff\\
\frac{-B-\sqrt{D}}{2A} &> 0 &\overset{\mathclap{\text{assuming $A > 0$}}}\iff\numberhere\label{eq:ymderivapos}\\
-B-\sqrt{4\left(y_1^2 - y_0y_2 + r_y(y_2 - 2y_1 + y_0)\right)} &> 0 & \iff\\
-\frac B2 - \sqrt{y_1^2 - y_0y_2 + r_y(y_2 - 2y_1 + y_0)} &> 0 & \iff\\
-(y_1-y_0)-\sqrt{y_1^2 - y_0y_2 + r_y(y_2 - 2y_1 + y_0)} &> 0 & \iff\\
y_0-y_1-\sqrt{r_y(y_0-2y_1)+(r_y-y_0)y_2+y_1^2} &> 0 & \iff\\
\sqrt{r_y(y_0-2y_1)+(r_y-y_0)y_2+y_1^2} &< y_1-y_0 & \overset{\mathclap{\text{assuming $y_1\geq y_0$}}}\iff\numberhere\label{eq:ymderivcond}\\
r_y(y_0-2y_1)+(r_y-y_0)y_2+y_1^2 &< y_0(y_0-2y_1)+y_1^2 &\iff\\
(r_y-y_0)(y_2-2y_1+y_0) &< 0 &\iff\\
r_y &< y_0&\numberhere\label{eq:ymderivend}
\end{align*}
In the last step it was used that $A=(y_2-2y_1+y_0)$ is assumed to be positive. We also see from \eqref{eq:ymderivcond} that if $y_1 < y_2$ then $t_m \centernot> 0$ (since the square root cannot be negative).

Otherwise if $A < 0$, we flip the inequalities (since we multiply by $A$ in \eqref{eq:ymderivapos}) and \eqref{eq:ymderivcond} gives:
\begin{equation*}
\sqrt{r_y(y_0-2y_1)+(r_y-y_0)y_2+y_1^2} > y_1-y_2
\end{equation*}
which is implied by $y_1 < y_2$. Otherwise, we can continue reducing this the same way as before, and we get
\begin{equation*}
(r_y-y_0)A > 0
\end{equation*}
which simply reduces to $r_y < y_0$.

To sum up, $t_m > 0$ can be determined by looking in the following table:
\begin{table}[H]
\centering
\begin{tabular}{c|c|c}
& $y_1 \geq y_0$ & $y_1 < y_0$ \\\hline
$A > 0$ & $r_y < y_0$ & False\\\hline
$A < 0$ & $r_y < y_0 $ & True
\end{tabular}
\end{table}
Or, succinctly:
\begin{equation*}
t_m > 0 \iff \big((y_1 \geq y_0) \land (r_y < y_0)\big) \lor \big((y_1 < y_0) \land (A < 0)\big)
\end{equation*}

In the same way, one can derive the conditions for $t_p > 0$:
\begin{align*}
t_p &> 0 & \iff\\
\frac{-B+\sqrt{D}}{2A} &> 0 &\overset{\mathclap{\text{assuming $A > 0$}}}\iff\numberhere\label{eq:ypapos}\\
-B+\sqrt{4\left(y_1^2 - y_0y_2 + r_y(y_2 - 2y_1 + y_0)\right)} &> 0 & \iff\\
-\frac B2 + \sqrt{y_1^2 - y_0y_2 + r_y(y_2 - 2y_1 + y_0)} &> 0 & \iff\\
-(y_1-y_0)+\sqrt{y_1^2 - y_0y_2 + r_y(y_2 - 2y_1 + y_0)} &> 0 & \iff\\
y_0-y_1+\sqrt{r_y(y_0-2y_1)+(r_y-y_0)y_2+y_1^2} &> 0 & \iff\\
\sqrt{r_y(y_0-2y_1)+(r_y-y_0)y_2+y_1^2} &> y_1-y_0 & \iff\numberhere\label{eq:ypextraatt}\\
r_y(y_0-2y_1)+(r_y-y_0)y_2+y_1^2 &> y_0(y_0-2y_1)+y_1^2 &\iff\\
(r_y-y_0)(y_2-2y_1+y_0) &> 0 &\iff\\
r_y &> y_0&\numberhere\label{eq:ypderivend}
\end{align*}
Here we pay special attention to \eqref{eq:ypextraatt}, since this is true if $y_1 \leq y_0$. If $y_1 < y_0$, we instead get the condition in \eqref{eq:ypderivend}. If $A < 0$ then the inequality is flipped at \eqref{eq:ypapos}, and we must have both $y_1 > y_0$ and $r_y < y_0$ for $t_p$ to be positive.

To sum up, the truth value of $t_p > 0$ can be determined using the following table:
\begin{table}[H]
\centering
\begin{tabular}{c|c|c}
& $y_1 \leq y_0$ & $y_1 > y_0$ \\\hline
$A > 0$ & True & $r_y > y_0$ \\\hline
$A < 0$ & False & $r_y > y_0$
\end{tabular}
\end{table}
Or more succinctly:
\begin{equation*}
t_p > 0 \iff \big((y_1 \leq y_0) \land (A > 0)\big) \lor \big((y_1 > y_0) \land (r_y > y_0)\big)
\end{equation*}

Now we have determined whether $t_m > 0$ and $t_p > 0$ and only need to check whether they are less than $1$. Let us for a moment try to look at $Q(1-t)$. By re-arranging we get
\begin{equation}
Q(1-t) = (\vec p_2 - 2\vec p_1 + \vec p_0)t^2 + 2(\vec p_1 - \vec p_2)t + \vec p_2
\end{equation}
which tells us that we can use the previous results to find the conditions where $1-t$ is positive (that is, where $t < 1$). We note that this curve is the same as $Q(t)$ when $\vec p_0$ and $\vec p_2$ has been exchanged.

\section{Other approaches}

\end{document}

